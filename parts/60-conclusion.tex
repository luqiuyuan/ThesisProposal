%======================================================================
\chapter{Conclusion}
\label{chap:c}
%======================================================================

Augmented reality is an emerging direction in industry and research, which has a broad use in various fields, e.g. games, communication, medicine and training. We focus on the use of AR in the area of training in this proposal. Manufacturers, such as Boeing, have already leveraged the advantages of augmented reality in their training department~\cite{caudell1992}. Great improvements have been made by using AR compared to traditional training resorts.

However, there are two main disadvantages with the existing AR training systems.
On one hand, pre-designed guidances, no matter they leverage an intelligent schema or not, are not suited for a wide range of different tasks, especially for the complex ones, since they are always designed as task-specified.
On the other hand, lack of collaboration between the trainer and the trainees limits the use of AR training methods in complex tasks, because the trainer is not able to get real-time feedbacks from the trainees.

We propose a novel framework for augmented reality training.
It has three main contributions:
\begin{itemize}
  \item
  We propose a method that allows the trainer to author guidances in real-time.
  \item
  We propose a method that enables the trainer to collect feedbacks from the trainees during the training process in real-time.
  \item
  We propose a method that makes the AR training framework work with devices without powerful graphics capacity.
\end{itemize}

The framework contains five parts: 1) video capture; 2) video object segmentation; 3) pose estimation; 4) remote rendering; 5) synchronization.
Among the five parts, we identify the three parts that we need to propose our own solutions: 1) video object segmentation; 2) remote rendering; 3) synchronization.

The method of video object segmentation is presented in Section~\ref{chap:vos}, which is used in the real-time object pose tracking system.
The proposed method is fully automatic, fast, and does not make restrictive assumptions about object motions.
In experiments on standard data sets, the proposed approach achieves comparable results to state-of-the-art video object segmentation methods, but our method is much faster.
It is an essential building block in a real-time region-based object pose tracking system.
We also demonstrate an application of the proposed method to content-aware video compression.

The work of remote rendering method is presented in Section~\ref{chap:hrr}.
Although the field of remote rendering has been studied for decades and the implementations have been used in many areas, such as gaming and virtual tour, researchers primarily focused on remote rendering of the entire frame.
We pay attention on the importance of models. In other words, we render object with higher importance with high-fidelity models, and render objects with lower importance with low-fidelity models.
It enables a trade-off between rendering quality and network delay.

The remaining work includes the implementation of the communication schema and the integration of all parts into one framework. which is described in Section~\ref{chap:dm}.

\section{Schedule}
\label{sec:c:s}

The following schedule is given in terms of the estimated time required following the completion of this proposal.

\textbf{Feb. to March. 2018 (2 months)}: Implement the communication schema and integrate all parts into one framework.

\textbf{Apr. 2018 to May. 2018 (2 months)}: Finish evaluating our proposed system.

\textbf{Jun. 2018 to Aug. 2018 (3 month)}: Finishing writing thesis.
